\documentclass[12pt]{article}
\usepackage{palatino}
\usepackage{graphicx}

\setlength{\textheight}{6in}
\setlength{\textwidth}{4.45in}
\usepackage[print]{booklet}
\source{\magstep0}{8.5in}{11in}
\target{\magstepminus1}{9.0in}{6.5in}
\setlength{\oddsidemargin}{1.5in}
\setlength{\evensidemargin}{1.5in}

\def\rulespace{0.64in}
\def\ruleshift{0.60in}
\def\ruleshiftright{0.10in}

\tolerance=5000

\newcounter{indented}
\newcounter{rulecount}[subsection]
\def\therulecount{\thesubsection.\arabic{rulecount}}
\def\rule{\refstepcounter{rulecount}
          \setcounter{subrulecount}{0}
              \filbreak
              \vskip 3pt plus 10pt
             \ifnum\value{indented}=0
                 \advance \leftskip by \rulespace
                 \advance \rightskip by \ruleshiftright
              \fi
              \value{indented} = 1
              \noindent \rlap{\hspace{-\ruleshift}\therulecount}}
\newcounter{subrulecount}[rulecount]
\def\thesubrulecount{\therulecount.\alph{subrulecount}}
\def\subrule{\refstepcounter{subrulecount}
              \filbreak
              \vskip 3pt plus 10pt
              \noindent \rlap{\hspace{-\ruleshift}\thesubrulecount}}


\let\sectionreal = \section
\def\section#1{\ifnum\value{indented}=1
                  \advance \leftskip by -\rulespace
                  \advance \rightskip by -\ruleshiftright
                  \value{indented}=0
               \fi \penalty-1500 \sectionreal{#1}}
\let\subsectionreal = \subsection
\renewcommand{\subsection}[2][*]{\ifnum\value{indented}=1
                     \advance \leftskip by -\rulespace
                     \advance \rightskip by -\ruleshiftright
                     \value{indented}=0
                  \fi \penalty-800 \subsectionreal{#2}
                     \if*#1 \else
                     \labeltab{#1}
                     \fi}

\def\thesection {\Roman{section}}
\newenvironment{segment}
               {\begin{tabular}{p{0.45in}p{3.70in}}}
               {\end{tabular} \vfill}
\renewcommand{\arraystretch}{1.3}

\def\asciiAlph#1{\ifnum0=\value{#1}@\else\Alph{#1}\fi}
\def\thesubsection {\Roman{section}.\asciiAlph{subsection}}

\newdimen{\tabspaceleft}
\tabspaceleft 6in
\def\bufferedtabs{}
\newcount\tabsproduced
\newcount\tabsproduced
\newcount\tabsconsumed
\def\labeltab#1{
   \advance\tabsproduced by 1
   \mark{\the\tabsproduced}
   \let\bufferedtabsold = \bufferedtabs
   \edef\bufferedtabs{\bufferedtabsold,\the\tabsproduced}
   \expandafter\newbox \csname bufferedtab\the\tabsproduced \endcsname
   \setbox\csname bufferedtab\the\tabsproduced \endcsname\vbox{#1}
}

\def\tabimagewrapper#1{\resizebox{!}{20pt}{\includegraphics{#1.eps}}}

\begin{document}
\def\vfudge{0.3in}
\def\hfudge{0.4in}
\def\tfudge{0pt}
%% Welcome to the true form of the Capture the Flag with Stuff Rules.
%% This is the base file from which the Html and TeX versions are generated.
%%
%% The format is largely inspired by wiki syntax, and fairly simple.
%% Hopefully it's obvious from usage, but the following is a complete
%% description to avoid confusion:
%%
%% ! name
%%    Major section "name".
%% 
%% !!{image} name
%%    Subsection "name".  "image" is optional; if given, the file image.eps
%%    will be used as the tab image for this subsection.
%%
%% * Rule ...
%%    A rule.  May stretch across multiple lines; ends at the next
%%    (sub)section or (sub)rule.
%%
%% ** Subrule ...
%%    A subrule; identical to a rule except for numbering purposes.
%%
%%   \label{label}
%%    After !, !!, *, or **, indicates the label of the section or rule for
%%    references.
%%
%% \ref{label}
%%    Indicates a reference to a labeled rule; will be substituted with the
%%    number of the section or rule associated with the label.  (Including
%%    only the levels necessary; e.g., a labeled subsection might show up as
%%    I.C, while a labeled subrule might show up as I.C.5.b)
%%
%% %Comment
%%    A comment until end of line.  Useful for documentation.
%%
%% \section = n
%% \subsection = n
%%    Set the current (sub)section counter.  Note that it is incremented
%%    before use (eh... it's how LaTeX does it), so setting it to -1 means the
%%    next subsection is subsection 0 (which the generated TeX will output
%%    @).  Only used once, but might as well make it general...
%%
%% @ Rule
%%    This is a special hack for rule 0.  It makes it section 0, and the HTML
%%    is a bit bigger than a normal section.
%%
%%  Text 
%% :*
%% :**
%%    Recent changes; this will show up as red in the HTML.  The first is for
%%    small changes, the second two for new rules/subrules
%%
%% Also, note that the conversions are a bit hackish and do basically no error
%% checking, so mistakes will likely manifest as either broken formatting or
%% bizarre TeX errors.  However, the format should be simple enough to avoid
%% them.

\section{Base Rules}

\setcounter{subsection}{-1}
\subsection[\tabimagewrapper{meta}]{Meta}
\rule When rules in this document conflict, the later rule takes
precedence.

\subrule E.g. (\ref{may_not_capture}) states that a captor may not capture, while
(\ref{doc_ock_may_capture})
states that a player wearing a Doc Ock's Belt may capture multiple times.

\rule The judges may choose to change the rules or add or remove stuff into
the game. Any such rules or stuff changes must be clearly explained before
the start of the game in which those changes are to be introduced.

\subsection[\tabimagewrapper{teams_territories}]{Teams and Territories}
\rule The game consists of two teams, red and yellow. Team membership is 
indicated by colored bands, worn on the upper arm or head, which must be
visible at all times.
\rule Each team has a home territory. One team's is Wean, the other's is Doherty.
The teams should swap territories after each game. The territory is defined
as the space inside the building, with the doors closed, or as defined by
the judges in the case of ambiguity. Where there is a chamber between two
sets of doors, the chamber is included in the building.
\rule Certain areas such as labs, clusters, and any normally
inaccessible or otherwise restricted areas are forbidden. No player may
enter these areas at any time during the game.
\rule All areas which are neither a team's home territory nor
forbidden are neutral. There must always be at least ten feet of neutral
territory between opposing teams' home territories.
\rule A player is considered to be in a team's territory if any
part of their body is in the team's territory. If a player is in no
team's territory, they are in neutral territory. Magic items have no
effect on players who are in neutral territory.

\subsection[\tabimagewrapper{players}]{Players}
\rule A player is any person playing the game.
\rule A teammate is a member of the same team.
\rule An enemy is a player who is not a teammate.
\rule A player may be normal, ethereal, captive, captor, prisoner,
interrogation victim, or stunned.
\rule A normal player is in the default state described in these rules.
\rule An ethereal player may not use, drop, or pick up magic items,
is not affected by magic items, may not capture, and may not be
captured. A player must use jazz hands to signal ethereality.

\rule A captive is a player who has been captured (see \ref{capture}).
Captives may not move or touch enemy flags and may not use or drop
   magic items (except for the Ninja Potion; see \ref{ninja}).
\rule A captor is a player who has captured another player (see
\ref{capture}). Captors must lead their captives to the jail (see \ref{capture}),
but are otherwise normal players.
\rule A prisoner is a player who was a captive, but has touched the Glyph of Jail
(see \ref{capture}).
\rule An interrogation victim is a prisoner who has been given truth serum
by the jailer (see \ref{capture}).  They may not use magic items, but otherwise act
identically to a prisoner except where specified.
\rule A stunned player must immediately sit on the floor in the territory in which they were
stunned until the stun wears off. The time they are stunned is measured from 
when they sit down, not when the stunning effect occurred. If they are captured
while stunned, they become unstunned. A player may not be stunned again while
stunned. Stunned players may not capture.
\rule Only normal players and captors can be stunned.
\subrule If John Mackey changes a player's major during the game, the player is
stunned for one minute.

\subsection[\tabimagewrapper{capture}]{Capture and Imprisonment} \label{capture}
\rule If a player touches an enemy when both are in the first
player's home territory, they may choose to capture the enemy. At this
point, the enemy becomes a captive.
\subrule As capture is optional, captors are encouraged to notify their
captives of this action ("Gotcha" is a good choice).
\rule A player, when captured, must immediately drop any flags they are carrying at 
the spot where they were captured.
\rule A captive or prisoner must give up any non-concealable items upon
request of the captor or a teammate of the captor. Concealable items need not
be given up unless the enemy asking for them knows that the player has the
item.
\rule  A captor must lead their captive directly to their team's Glyph \label{may_not_capture}
of Jail (see \ref{glyph_of_jail}) at a reasonable speed. They cannot make any more
captures until their captive has entered jail by touching the Glyph of
Jail or another prisoner in the jail.  Once this
occurs, the captor is again a normal player, and the captive becomes a
prisoner.
\rule If a captor is stunned, their captive becomes a normal player.
\rule Prisoners may use magic items.

\subsection{ Jail} \label{jail}
\rule Each team has a jail, designated by the Glyph of Jail.
\subrule Whenever possible, the Glyph of Jail in Doherty will be placed on floor A, 
near the staircase closest to the main lobby, and the Glyph of Jail in Wean 
will be placed on floor 6, near room 6220 at the end of the 6200 corridor.
\rule One player on each team may take on the role of jailer by
wearing the team's Jailer Fedora. The jailer is a normal player except
as described in the remainder of this section. The jailer may at any
time transfer the fedora to a teammate, who becomes jailer, or
remove it, leaving no jailer.
\subrule In particular, if the jailer is captured or stunned, the jail
remains functional; prisoners are not released (although any interrogation 
in progress is ended).
\rule A player may not move the opposing team's fedora except to drop
it.
\rule The prisoners must form one or more chains, with each player
touching the Glyph of Jail either directly or through a chain of other
prisoners.
\rule If there is a jailer, they may relax the requirement to form chains.
\subrule Typically, the jail is at the end of a hall, and prisoners are allowed to 
sit around in the vicinity of the jail.
\rule  Every fifteen minutes, starting from the beginning of play, \label{jailbreak}
all prisoners are released from all jails. Released prisoners must
return to their home territories at reasonable speed, by the most
direct route. Prisoners released in this manner are ethereal until they
re-enter their home territories.
\rule A player in a normal state not carrying a flag may touch a prisoner
in another team's jail. As long as freeing the prisoner does not break a
chain leading back to the jail, the prisoner is freed. Both players are
ethereal and must return to their home territories by the most direct
route; they need not remain together.
\rule The jailer has a supply of Truth Serum, which they may use by
poking a prisoner in the shoulder with a pinky finger. The prisoner
then becomes an interrogation victim.
\rule The jailer may then ask the victim up to six yes/no
questions, and the victim may lie no more than once. The victim may
answer "I don't know" if they really don't know, but then the question
doesn't count against the six.
\rule Upon completion of questioning, the victim is freed from jail
and must return to their home territory as after a jailbreak (see \ref{jailbreak}).
Questioning is complete when the victim answers the sixth question, if
the jailer decides they are done with this victim, if the victim is
freed, at jailbreak, if there is no jailer, or if the jailer leaves line of
sight of the victim.
\rule There may be at most one truth serum victim in each jail.

\subsection[\tabimagewrapper{setup}]{Initial Setup and Game Completion}
\rule At the beginning of the game, there is a setup period of
fifteen minutes. This time should be used for the placement of flags
and glyphs, and distribution of magic items. No player may enter enemy
territory during this period. Magic items have no effect during this
time.
\rule The game begins at the end of the set up period and ends after one
hour, or when there remains only one team with uncaptured flags.

\subsection[\tabimagewrapper{terms}]{Additional Terms}
\rule Cooldown: This is a property of certain items. An item with
cooldown may not be activated again for one minute after its use. This
requirement may be suspended by the singing of a number of verses from
a song determined before the game begins; the item is then usable after
singing that song.
\rule Concealment: Potions may be concealed, meaning they need not
be visible. All other game items must be visible at all times. A
captive is not required to give up a concealed item unless their captor
asks for it specifically and knows that they have the item.
\subrule Potions cannot be used while concealed.
\rule Dispelling: A dispelled item has no magical effects.
\rule Field of View: The field of view is the extent of the observable world that
is visible to the player at any given moment.
\rule Line of Sight: A player is within line of sight of an item or event
if they can rotate themself to bring it into their field of view.
\rule See: A player sees an item or event if it passes within their field of
view and they can recognize it.
\rule Sacrificial: Certain items are sacrificial. These items may
only be activated in enemy territory, must be dropped upon use, and may not
be picked up or moved when in enemy territory.
\rule Whap: A player is whapped by a wand if it hits them at approximately the same
time the wand bearer shouts the key word.
\subrule As wands are fairly soft, whapping a player on the head counts but is discouraged.
\rule Key Words/Phrases: Some items have a key word or phrase used to
activate them.  This phrase must be said entirely in the territory where the
activation occurs, and the action occurs upon completion of the phrase.
\rule Knowledge: knowledge of an opponent's concealable items, as referenced in 
rules I.C.3 and I.F.2, means that they have not had a chance to dispose of 
that item since the last time you saw it in their possession.


\subsection[\tabimagewrapper{flags}]{Flags}
\rule Flags are large pieces of cloth or felt. Each team has three or more flags.
\rule Flags must be placed during the initial setup time, and their
placement must be supervised by a judge. They must be placed no higher
than six feet above the floor; the flag must be clearly visible from most
angles; flags must be placed no more than one flag per
hallway or corridor. If the flag is placed inside a container, there
must be an obvious way to remove the flag in one smooth motion.
\rule No flag may ever exist in an area with fewer than two exits,
unless it is being carried. If a flag is dropped in an area with only
one exit, a judge must move it the shortest possible distance to a
valid location.
\rule Any magic items a player is carrying are dispelled if they are
carrying a flag, and remain dispelled for one minute after the player
gives up the item or drops the flag.
\rule Each flag will be assigned a point value at the beginning of the game by the
head judge.
\subrule Judges may or may not inform the players of these values on a per-game basis.
\rule The winner of the game is the team with the most points. The point values of
each flag must be set such that the team with the most flags has the most points.
\subrule If the point system is set differently than specified in 1.G.6, the players
must be informed of this fact before the set up of the game in which this takes place.
\rule A player who brings a flag to the judges room earns for their
team the number of points associated with the flag. The flag is then
out of play for the remainder of the game.

\subsection[\tabimagewrapper{judges}]{Judges}
\rule Judges are indicated by blue or silver headbands, and are responsible for
keeping the game running smoothly.
\rule The head judge is the final arbiter and scorekeeper, and should remain in the
judges' room, which is Wean Hall 8427 unless otherwise specified.
\rule An assistant head judge should reside at each team's jail.
\rule Each team should also have at least 1 roaming judge. The total number of
roaming judges is up to the head judge, and based on the number of players.
\subrule Roaming judges’ responsibility is to resolve conflicts or perform other
duties required of them.
\rule Judges should attempt to resolve disputes fairly and find outcomes agreeable
to both sides. In all cases, the judge's decision is final.


\section{Glyphs}

\subsection[\tabimagewrapper{glyphs}]{General}
\rule Glyphs are large sheets of paper or poster-board marked with a spell.
\rule Each glyph is owned by a particular team, indicated by the coloring of the glyph.
\rule A glyph must be taped to a wall in the owning team's territory or side
 of a door that would be in the owning team's territory if it were closed.
\subrule Once placed, all glyphs are considered part of the owning team’s territory.
\rule A glyph affects all normal players on opposing teams in its owning team's
  territory.
\rule A glyph may not be placed within ten feet of a flag.  If a flag is dropped
  within ten feet of a dispellable glyph, the glyph is dispelled for as long as
  the flag remains, and for one minute after.

\subsection[\tabimagewrapper{jail}]{Glyph of Jail} \label{glyph_of_jail}
\rule The Glyph of Jail is indicated by the word "Jail".
\rule The Glyph of Jail defines the Jail, as described by section \ref{jail}.
\rule The Glyph of Jail cannot be dispelled.

\subsection{Glyph of Entrancement}
\rule The Glyph of Entrancement is indicated by the word "Gotcha".
\rule A player who sees the Glyph of Entrancement is stunned for one minute.

\subsection{Glyph of the Disgusting Doorknob}
\rule The Glyph of the Disgusting Doorknob is indicated by the word "Yukko".
\rule If a Glyph of the Disgusting Doorknob is on a door, a player may not
open the door or prevent it from closing by interacting with the side of
the door the Glyph is on. If the Glyph is mounted on one of a set of doors,
all doors are affected.
\subrule The Glyph of Disgusting Doorknob affects players in neutral territory.
\rule The counting of exits for purposes of flag placement is unaffected by
this glyph.
\rule If a Glyph of the Disgusting Doorknob is mounted next to the
buttons controlling an elevator, no affected player may push any of the
buttons or prevent the elevator doors from closing.
\rule Any player must be able to exit to neutral territory without opening any
door from a side affected by a Glyph of the Disgusting Doorknob or activating 
any stuff. No player can be completely trapped by this glyph.

\subsection{Glyph of Alarm}
\rule The Glyph of Alarm is indicated by the word "Alarm".
\rule A player who sees the Glyph of Alarm must shout or sing
"Alarm" or another phrase as loudly as possible for one minute or
until they are captured or returns to neutral space.


\section{Potions}

\subsection[\tabimagewrapper{potions}]{General}
\rule Potions are small pieces of foam wrapped in colored tape.
\rule Potions are concealable.

\subsection{ Ninja Potion of Magic Smoke} \label{ninja}
\rule The Ninja Potion of Magic Smoke is red and sacrificial.
\rule To activate this item a captive throws this potion onto the ground and yells
as loudly as possible "Poof, I am a Ninja!". All enemy players who hear this
and are within line of sight of the user are stunned for ten seconds.
\subrule This includes enemy players within line of sight in all
territories. Players in neutral territory are unaffected, as they are
immune to all stuff.

\subsection{Potion of Wait... There's a Key at the Bottom of This}
\rule The Potion of Wait... There's a Key at the Bottom of This is green and has cooldown.
\rule To activate this potion, a player must bonk themself on the head
with this potion and use one of the key words "Lolt", "Jube", or "Key".
\rule A stunned player, once they are sitting, may use this potion to unstun 
themself and return to a normal state.
\rule A prisoner may use this potion to free themself from jail and return to a normal state.

\subsection{Potion of Mind Control}
\rule The Potion of Mind Control is blue.
\rule Only captors may use the Potion of Mind Control. To use this potion, a 
captor must hand the potion to their captive and use the key word "obey."
\rule A captive who is handed this potion must proceed directly to jail as if 
escorted by the captor, clearly displaying the potion to any members of the 
captor's team they encounter.
\subrule A player controlled by the Potion of Mind Control is still a captive. They 
may not be stunned, and they ignore opponents' glyphs.
\subrule No player may take the Potion of Mind Control from a captive being 
controlled by it, even if they know they have it.
\subrule If the Potion of Mind Control is dispelled, the captive under its control 
is freed.
\rule A captor who uses the Potion of Mind Control is then stunned for 10 seconds.
After that, they are again a normal player. They do not have to escort the 
captive to jail.
\rule The captive may not proceed to jail by a route that is less than direct, in 
an effort to explore for flags or for any other reason.


\section{Wands}

\subsection[\tabimagewrapper{wands}]{General}
\rule Wands are eighteen-inch lengths of foam rubber.
\rule The effect of a wand is activated by whapping a person or glyph and shouting
the key word.
\rule Each use of the key word invokes at most one wand on one subject.

\subsection{Wand of Vengeance}
\rule The Wand of Vengeance has key word "Toast", and is red and sacrificial.
\rule A normal player may whap an enemy in the enemy's home territory.
The user captures the enemy, and both players become ethereal.  The user
must lead the enemy back to their jail as directly as possible at reasonable
speed.
\rule Upon entry into the user's home territory, the user becomes a captor and the
enemy becomes a captive.

\subsection{Wand of Stun}
\rule The Wand of Stun has key word "Stun", and is green and has cooldown.
\rule Any player whapped with this wand is stunned for one minute.

\subsection{Wand of Dispel}
\rule The Wand of Dispel has key word "Dispel", and is blue and has cooldown.
\rule If a player is whapped with the wand of dispel, all magic items they are
carrying are dispelled for one minute.
\rule If two players simultaneously use this wand on each other,
all items carried by both players are dispelled for one minute.
\rule A dispellable glyph which is whapped by the Wand of Dispel is dispelled for
one minute.


\section{Belts}

\subsection[\tabimagewrapper{belts}]{General}
\rule A belt is a long sash of cloth or felt.
\rule To be used, a belt must be tied around a player's waist.
\rule A belt can be worn by at most one player.
\rule A player can wear at most one belt.

\subsection{Andy Warhol's Belt of Pop Occultism}
\rule Andy Warhol's Belt of Pop Occultism is red and sacrificial.
\rule To activate this belt, the wearer must yell as loudly as
possible some phrase to be determined by the judges before the start of
the game. Upon activation, all enemy players within line of sight of
the user are stunned for one minute.
\subrule Since the belt is sacrificial, the user must also drop the belt.

\subsection{Goombah's Belt of Humiliating Protection}
\rule Goombah's Belt of Humiliating Protection is green.
\rule This belt is activated by starting to skip and loudly sing "Yankee Doodle".
While singing and skipping continues, the belt is in use, and the wearer is
immune to capture and stun.
\rule Any number of teammates may share the protection of this belt
by holding hands with the wearer, either directly or through a chain of
teammates, as long as all teammates in the chain are skipping and
singing along.
\rule If any player in the chain is dispelled, all items carried by all members of the chain are dispelled.
\subrule In particular, the Goombah's belt is dispelled.
\rule This belt does not function in elevators.

\subsection{Doc Ock's Belt of Many Arms}
\rule Doc Ock's Belt of Many Arms is blue.
\rule  The wearer may have up to four captives, as long as they \label{doc_ock_may_capture}
are within an arm's length of each previous captive at the time of the next
enemy's capture.
\subrule Once any captures have been made, they must still lead the
captives back to jail as usual, but may still capture on the
way to jail according to (\ref{doc_ock_may_capture}).
\rule If the belt is dispelled, all of its wearer's captives except the first are
freed. If there is a tie, the captor may choose which one to keep.

\section{Sportsmanship}

\subsection[\tabimagewrapper{sportsmanship}]{Physicality}
\rule Players may not physically impede or block other players, either with their
bodies or with objects.
\subrule This includes actions like preventing elevators from
moving. Breakage of this rule is grounds to eject you from the game.
\rule You may not change official timekeeping watches.
\rule You may not fabricate or replicate game items.

\subsection{Lies}
\rule Although lies and deceit are part of the game, there are several things you
may not conceal, obfuscate, or lie about.
\rule You may not lie about whether you're in the game.
\rule You may not lie about which team you're on.
\rule You may not lie about the status of the game.
\subrule For example, whether the game has started or ended.
\rule You may not lie about what non-concealable items you have.
\rule You may not lie about cooldown or dispel status of items.
\rule You may not lie about any player's status.
\rule You may not make jazz hands unless you are ethereal.
\rule You may not display a Potion of Mind Control in enemy territory unless you
are a captive under its control.
\rule You may not imitate the activation or use of an item without the intent of
using it.

\def\thesection {\arabic{section}} \setcounter{section}{-1} \section{Respect: Above all, don't be a jackass.}
\end{document}
